% Answer for http://tex.stackexchange.com/q/245691/828
\documentclass{article}
\usepackage{graphicx}
\usepackage{tikz}

\newlength\imagewidth
\newlength\imagescale

\begin{document}

\begin{figure}
	\centering
	\includegraphics[width=0.618\linewidth]{img1}
	\caption{image 1}
\end{figure}

\begin{figure}
	\centering
	\includegraphics[width=0.309\linewidth]{img2}
	\caption{image 2}
\end{figure}

\begin{figure}
	\centering
	\pgfmathsetlength{\imagewidth}{\linewidth}%
	\pgfmathsetlength{\imagescale}{\imagewidth/524}%
	\begin{tikzpicture}[x=\imagescale,y=-\imagescale]
		\node[anchor=north west] at (0,0) {\includegraphics[width=\imagewidth]{img1}};
		\node[anchor=north west] at (300,100) {\includegraphics[width=0.25\imagewidth]{img2}};
	\end{tikzpicture}
	\caption{Both images on top of each other}
\end{figure}

\end{document}
