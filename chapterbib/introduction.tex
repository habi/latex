%!TeX root = main.tex
\chapter{Introduction}
To make this document work, you have to supply your own bibliography file in
each chapter (change {libraby.bib} on the last line to your \BibTeX-file) and
change the \verb+\cite+-commands to keys present in your bibliography.

Then 
\begin{itemize}
\item run \verb+pdflatex+ on the main file
\item run \BibTeX\ on each of the \verb+.aux+ files generated for each chapter.
\item run \verb+pdflatex+ on the main file
\item run \verb+pdflatex+ on the main file
\end{itemize}

You should be able to achieve all of the points above with the tiny bash script
shown below, which does \verb+pdftex+ the main file a sufficient amount of times
and calls \BibTeX\ for each of the \verb+.aux+ files found in the directory it
is run in between those calls.

\lstinputlisting[language=bash]{compilationscript.bash}

You can enter these commands one after one in your terminal, or just call
\verb+bash compilationscript.bash+ in your terminal if you're using some kind of
UNIX system.

\cite{Author2013}
\bibliographystyle{plainnat}
\bibliography{library.bib}
