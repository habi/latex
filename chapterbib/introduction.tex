%!TeX root = main.tex
\chapter{Introduction}
The document here uses the an example bibliography (\directory{/usr/share/texlive/texmf-dist/bibtex/bib/base/xampl.bib})\footnote{At least that's where the file is on Ubuntu 16.04}, based on \href{https://tex.stackexchange.com/a/57599/828}{this tex.SE} question to show how to make a bibliography for each chapter of a big document.

To typeset it
\begin{itemize}
	\item run \verb+pdflatex+ on the main file
	\item run \hologo{BibTeX} on each of the \verb+.aux+ files generated for each chapter.
	\item run \verb+pdflatex+ on the main file
	\item run \verb+pdflatex+ on the main file
\end{itemize}

You should achieve the same result with the tiny bash script below, which \verb+pdftex+es the main file a sufficient amount of times and calls \hologo{BibTeX} for each of the \verb+.aux+ files inbetween those calls.

\lstinputlisting[language=bash]{compilationscript.bash}

You can enter these commands one after one in your terminal, or just call \verb+bash compilationscript.bash+ in your terminal if you're using some kind of UNIX system.

And here is a very important citation~\cite{article-full}, which is what this is all about\ldots
\bibliographystyle{plainnat}
\bibliography{xampl}
